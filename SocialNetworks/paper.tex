% This will be the main document for the Social Networks paper to
% be written by the Eggnet team of Jordan Ell, Triet Huynh and Braden
% Simpson in association with Adrian Schroeter and Daniela Damian.

\documentclass[conference]{IEEEtran}

% Correct bad hyphenation here
\hyphenation{op-tical net-works semi-conduc-tor}

% Begin the paper here
\begin{document}


% Paper title
% Can use linebreaks \\ within to get better formatting as desired
\title{Linking Communication Artifacts to Changesets}

% Author names
% Use a multiple column layout for up to three different affiliations
\author{\IEEEauthorblockN{Jordan Ell}
\IEEEauthorblockA{University of Victoria\\
Victoria, British Columbia\\
jell@uvic.ca}
\and
\IEEEauthorblockN{Triet Huynh}
\IEEEauthorblockA{University of Victoria\\
Vancouver, British Columbia\\
infiro@uvic.ca}
\and
\IEEEauthorblockN{Braden Simpson}
\IEEEauthorblockA{University of Victoria\\
Victoria, British Columbia\\
braden@uvic.ca}}

% make the title area
\maketitle

\begin{abstract}
As software systems get more complex, the companies developing them consist of larger teams and therefore produce more complex communication artifacts.  To find out how to mitigate losses created by this growth and complexity, companies need to work harder to find out more efficient and effective ways to communicate.  This paper provides an analysis of communications artifacts, linked to commits by using multiple methods of textual analysis on the artifact, as well as the changeset.  These methods result in links that provide ways to analyze these communication artifacts and their participants to produce a historical representation in the form of patterns, which can be used to predict outcomes based on participants and communication.
\end{abstract}


\section{Introduction}
This demo file is intended to serve as a ``starter file''
for IEEE conference papers produced under \LaTeX\ using
IEEEtran.cls version 1.7 and later.
I wish you the best of success.

\hfill mds
 
\hfill January 11, 2007

\subsection{Subsection Heading Here}
Subsection text here.

\subsubsection{Subsubsection Heading Here}
Subsubsection text here.


\section{Conclusion}
The conclusion goes here.


\section*{Acknowledgment}
The authors would like to thank...


\begin{thebibliography}{1}

\bibitem{IEEEhowto:kopka}
H.~Kopka and P.~W. Daly, \emph{A Guide to \LaTeX}, 3rd~ed.\hskip 1em plus
  0.5em minus 0.4em\relax Harlow, England: Addison-Wesley, 1999.

\end{thebibliography}


% End of the paper
\end{document}
