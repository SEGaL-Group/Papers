% This will be the main document for the Technical Networks paper to
% be written by the Eggnet team of Jordan Ell, Triet Huynh and Braden
% Simpson in association with Adrian Schroeter and Daniela Damian.

\documentclass[conference]{IEEEtran}

% Correct bad hyphenation here
\hyphenation{op-tical net-works semi-conduc-tor}

% Begin the paper here
\begin{document}


% Paper title
% Can use linebreaks \\ within to get better formatting as desired
\title{Changeset Based Technical Dependencies}

% Author names
% Use a multiple column layout for up to three different affiliations
\author{\IEEEauthorblockN{Jordan Ell}
\IEEEauthorblockA{University of Victoria\\
Victoria, British Columbia\\
jell@uvic.ca}
\and
\IEEEauthorblockN{Triet Huynh}
\IEEEauthorblockA{University of Victoria\\
Vancouver, British Columbia\\
infiro@uvic.ca}
\and
\IEEEauthorblockN{Braden Simpson}
\IEEEauthorblockA{University of Victoria\\
Victoria, British Columbia\\
braden@uvic.ca}}

% make the title area
\maketitle


\begin{abstract}
Software systems have not only become larger themselves over time, but the amount of
technical contributors and dependencies have also increased. With these expansions, also comes
the increasing risk of introducing fault causing or fix inducing software into a pre existing system.
This paper describes how evolving analysis of method caller and callee relationships can be
accomplished to give network centric data of Java based applications. The result of these networks
are method call graph relationships per changeset in a software configuration managements (SCM) 
repository. These networks can be used to analyze the project impact of a changeset on a
technical contributor level.
\end{abstract}


\section{Introduction}
This demo file is intended to serve as a ``starter file''
for IEEE conference papers produced under \LaTeX\ using
IEEEtran.cls version 1.7 and later.
I wish you the best of success.

\hfill mds
 
\hfill January 11, 2007

\subsection{Subsection Heading Here}
Subsection text here.

\subsubsection{Subsubsection Heading Here}
Subsubsection text here.


\section{Conclusion}
The conclusion goes here.


\section*{Acknowledgment}
The authors would like to thank...


\begin{thebibliography}{1}

\bibitem{IEEEhowto:kopka}
H.~Kopka and P.~W. Daly, \emph{A Guide to \LaTeX}, 3rd~ed.\hskip 1em plus
  0.5em minus 0.4em\relax Harlow, England: Addison-Wesley, 1999.

\end{thebibliography}


% End of the paper
\end{document}
