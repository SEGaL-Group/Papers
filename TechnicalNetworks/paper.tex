% This will be the main document for the Technical Networks paper to
% be written by the Eggnet team of Jordan Ell, Triet Huynh and Braden
% Simpson in association with Adrian Schroeter and Daniela Damian.

\documentclass[conference]{IEEEtran}

% Correct bad hyphenation here
\hyphenation{op-tical net-works semi-conduc-tor}

% Begin the paper here
\begin{document}


% Paper title
% Can use linebreaks \\ within to get better formatting as desired
\title{Changeset Based Technical Dependencies}

% Author names
% Use a multiple column layout for up to three different affiliations
\author{\IEEEauthorblockN{Jordan Ell}
\IEEEauthorblockA{University of Victoria\\
Victoria, British Columbia\\
jell@uvic.ca}
\and
\IEEEauthorblockN{Triet Huynh}
\IEEEauthorblockA{University of Victoria\\
Vancouver, British Columbia\\
infiro@uvic.ca}
\and
\IEEEauthorblockN{Braden Simpson}
\IEEEauthorblockA{University of Victoria\\
Victoria, British Columbia\\
braden@uvic.ca}}

% Make the title area
\maketitle


\begin{abstract}
Software systems have not only become larger over time, but the amount of
technical contributors and dependencies have also increased. With these expansions, also comes
the increasing risk of introducing a software failure into a pre existing system.
Software failures are a multi billion dollar problem in the industry today and while integration and
other forms of testing are helping to ensure a minimal number of failures, research to understand
full impacts of changesets and their social implications is still a major concern. This paper describes
how analysis of changesets and the social impacts they infer can be used to detect when failures 
may occur and between what contributors the failures are being caused by. The result of this
analysis comes in the form of a communication recommendation system. Based on previous
changeset technical dependencies, their social implications and the build success or failure, 
we will be able to recommend that a pair of contributors should communicate to avoid a potential
software failure.
\end{abstract}


\section{Introduction}
Often, object oriented programming (OOP) languages are used to create highly modular and 
reusable code. The problem lies in that objects and their internal methods can be used in a wide
variety of locations in either large or small projects. This causes small or large changes to any 
given method inside the project to have a rippling effect across the rest of the project. These 
effects can become a large influence of software failures throughout the project's life span.

As projects evolve over time, they tend to become larger and more complex in nature and as
a result the amount of technical dependencies will increase for a given module of code.
Studies have shown that the more technical dependencies that a changed section of code has,
the greater the number of software failures introduced will be \cite{Zimmermann:2008:PDU}. 
This knowledge opens the door to many types of network analysis in regard to technical dependencies.

Other researchers, such as Pinzger et al. \cite{Pinzger:2008:DNP} and Schroter et al.
 \cite{Schroter:2010:PBO}, as well as Zimmermann\cite{Zimmermann:2008:PDU}, have also shown that
social and communication networks can be used to predict a changeset's possibily of failure or
success.

Pinzger and Zimmermann use the idea of determining types of social connectivity through 
software repository mining. They extract technical dependencies on the binary module level and
determine what contributors have work on the module for a give changeset to create the social
contributor networks. From these networks they are able to predict the success or failures of
builds based on the notion of socio-technical congruence as described by Cataldo \cite{Cataldo:2006:ICR}.
Schroter, on the other hand, uses issue or bug tracking systems to create a social network of
contributors who communicate around such an item. Technical edges are then added to the
nodes of contributors if both have edited the same file for a given build.

We intend to use a similar approach of extracting social networks from technical dependencies, 
except on the method or function level. From these dependencies we can extrapolate weighted
social relationships between developers who are involved in the technical dependencies. Through
pattern mining of these networks over a project's life span, we will be able to recommend future
communication between contributors given a changeset in order to avoid potential software
failures.


\section{Data Collection}

\subsection{Preliminary Data}
Subsection text here.


\section{Methodology}
Describe our methodology.

\subsection{Extracting Technical Networks}
Subsection text here.

\subsection{Extracting Technical Networks}
Subsection text here.

\subsection{Extracting Technical Networks}
Subsection text here.

\subsection{Extracting Technical Networks}
Subsection text here.


\section{Results}
Describe our results.


\section{Conclusion}
Describe our conclusion.


\section*{Acknowledgment}
The authors would like to thank...


\bibliographystyle{IEEEtran}
\bibliography{paper}


% End of the paper
\end{document}
