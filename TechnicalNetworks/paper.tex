% This will be the main document for the Technical Networks paper to
% be written by the Eggnet team of Jordan Ell, Triet Huynh and Braden
% Simpson in association with Adrian Schroeter and Daniela Damian.

\documentclass[conference]{IEEEtran}

% Correct bad hyphenation here
\hyphenation{op-tical net-works semi-conduc-tor}

% Begin the paper here
\begin{document}


% Paper title
% Can use linebreaks \\ within to get better formatting as desired
\title{Changeset Based Communication To Prevent Software Failures}

% Author names
% Use a multiple column layout for up to three different affiliations
\author{\IEEEauthorblockN{Jordan Ell}
\IEEEauthorblockA{University of Victoria\\
Victoria, British Columbia\\
jell@uvic.ca}
\and
\IEEEauthorblockN{Triet Huynh}
\IEEEauthorblockA{University of Victoria\\
Vancouver, British Columbia\\
triethuynhm@gmail.com}
\and
\IEEEauthorblockN{Braden Simpson}
\IEEEauthorblockA{University of Victoria\\
Victoria, British Columbia\\
braden@uvic.ca}}

% Make the title area
\maketitle


\begin{abstract}
Software systems have not only become larger over time, but the amount of
technical contributors and dependencies have also increased. With these expansions, also comes
the increasing risk of introducing a software failure into a pre-existing system.
Software failures are a multi billion dollar problem in the industry today and while integration and
other forms of testing are helping to ensure a minimal number of failures, research to understand
full impacts of changesets and their social implications is still a major concern. This paper describes
how analysis of changesets and the social relationships they infer can be used to detect when failures 
may occur and between what contributors the failures are induced. The result of this
analysis comes in the form of a communication recommendation system to prevent possible software 
failures. Based on previous changeset technical dependencies, their social implications and their 
success or failure,  we will be able to recommend that a pair of contributors communicate
to avoid a potential software failure.
\end{abstract}


\section{Introduction}
Often, object oriented programming (OOP) languages are used to create highly modular and 
reusable code. The problem lies in that objects and their internal methods can be used in a wide
variety of locations in either large or small projects. This causes small or large changes to any 
given method inside the project to have a rippling effect across the rest of the project. These 
effects can become a large influence of software failures throughout the project's life span.

As projects evolve over time, they tend to become larger and more complex in nature and as
a result the amount of technical dependencies will increase for a given module of code.
Studies have shown that the more technical dependencies that a changed section of code has,
the greater the number of software failures introduced will be \cite{Zimmermann:2008:PDU}. 
This knowledge opens the door to many types of network analysis in regard to technical dependencies.

Other researchers, such as Pinzger et al. \cite{Pinzger:2008:DNP} and Schroter et al.
 \cite{Schroter:2010:PBO}, as well as Zimmermann\cite{Zimmermann:2008:PDU}, have also shown that
social and communication networks can be used to predict a changeset's possibily of failure or
success.

Pinzger and Zimmermann use the idea of determining types of social connectivity through 
software repository mining. They extract technical dependencies on the binary module level and
determine what contributors have work on the module for a give changeset to create the social
contributor networks. From these networks they are able to predict the success or failures of
builds based on the notion of socio-technical congruence as described by Cataldo \cite{Cataldo:2006:ICR}.
Schroter, on the other hand, uses issue or bug tracking systems to create a social network of
contributors who communicate around such an item. Technical edges are then added to the
nodes of contributors if both have edited the same file for a given build.

We intend to use a similar approach of extracting social networks from technical dependencies, 
except on the method or function level. From these dependencies we can extrapolate weighted
social relationships between developers who are involved in the technical dependencies. Through
pattern mining of these networks over a project's life span, we will be able to recommend future
communication between contributors given a changeset in order to avoid potential software
failures.


\section{Data Collection}
For this paper, we choose to study the Hibernate-ORG project which is an open source Java 
application hosted on GitHub\footnote{https://github.com/}. 
The issue tracking for this project is preformed by Jira\footnote{http://www.atlassian.com/software/jira/overview}.

Our tool heavily relies on a project being source controlled by Git as we make use of many
git features to be able to traverse and mine information from the repository. Our tool also
requires issue tracking software for the project to be able to mine for changeset success or
failure status.


\section{Methodology}
In the following, we describe how we were able to predict potential software failures using changeset
technical dependencies.

\subsection{Extracting Technical Networks}
To extract the technical dependencies of a project given a changeset, we construct a method
call graph. Unlike  Bodden's \cite{Bodden:2003:HVJ} and other's approaches of using byte code
and whole projects, we built our call graphs using source code, which does not have the assumptions
of being able to compile or have access to all project files, using Eclipse's ASTParser. 
With the use of source files, we are able
to update the call graph as we traverse over the repository dynamically as opposed to rebuilding
at every snapshot of the project. The resulting call graphs contains information regarding the control
flow of methods inside of the project at every given changeset.

We then use a tool similar to the Unix diff command to figure out what changes we made to any given
file inside of the changeset in question. From here we can determine the methods, and weight
of each method in terms of total characters , that have been changed.

Combining the call graph and the given method changes of the changeset from the diff like command,
we can determine what methods call those which have been changed. This creates a technical
dependency between the two methods. We can also create a depth level of method calls to determine
the technical dependencies between a daisy chain of method calls that end in the changed methods.

\subsection{Extracting Contributor Networks}
From the method caller to callee relationships, we can infer the contributor relationships by using
the git blame command. From the blame, we can gather information regarding which contributor 
wrote which lines of code inside a method results in their overall contribution to any given method.
We now gather all contributors of the caller method inside the technical relationship and create an
edge between them and the author of the changeset, who is the contributor for the callee method.
This creates a contributor network for a given changeset. To weight the edges, we use the formula:

\begin{equation}\label{first}
W = (\frac{\frac{\delta}{T_{1}}*\frac{O}{T_{2}}}{D})*100
\end{equation}

Here, $\frac{\delta}{T_{1}}$ is the percent of method changed by the callee and $\frac{O}{T_{2}}$ is the percent
owned of the caller method. All of this is divided by the caller depth away from the callee (most cases
1). 

\subsection{Determining Changeset Success or Failure}
In order to determine whether a previous commit was a success or failure (introduces a software
failure), we use the same approach as Sliwerski et al. \cite{Sliwerski:2005:CIF} in referencing the project's issue tracking
database for additional information. However, changes were made in order to make their syntactic 
and semantic analysis more project independent as their project was only used on specific Java projects. 

Each commit from the project's repository will now be labelled as a success
or failure and will be used for pattern mining in the prediction subsection below.

\subsection{Predicting Success or Failure}
Given a new changeset to a project, we are able to extract all the contributor networks as described
above. After which, we look into the project's past to determine if any of the contributor edges have
occurred before and whether or not that edge corresponded to a successful or failed changeset.

If the pattern in question is found to be more associated with past failed changesets, we can predict that
the current changeset will possibly fail and that to prevent this failure, the two contributors of the 
pattern should communicate in the hopes of catching software failures earlier. If however, the pattern
is found to be more associated with successful changesets, we can predict that the current changeset
will be successful and not require a communication recommendation between contributors.


\section{Results}
Describe our results.


\section{Conclusion and Future Work}
Describe our conclusion.


\section*{Acknowledgment}
The authors would like to thank yo' mama.


\bibliographystyle{IEEEtran}
\bibliography{paper}


% End of the paper
\end{document}
